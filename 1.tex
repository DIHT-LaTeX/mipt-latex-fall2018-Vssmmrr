\documentclass[12pt]{article} 

\usepackage[russian]{babel} 

\title{Домашнее задание №1} 
\author{Полина Хохонова} 
\date{} 

\begin{document} 
   \maketitle 
   \begin{flushright} 
   {\itshape Audi multa,\\ loquere pauca} 
   \end{flushright} 
   \hspace{20 pt} 
   \begin{center} 
     Это мой первый документ в системе компьютерной вёрстки \LaTeX
   \end{center}
   \begin{center}
   \large \sffamily <<Ура!!!>> 
    \end{center}
    А теперь формулы. {\scshapeФормула}~--- краткое и точное словесное выражение, определение или же ряд математических величин, выраженный условными знаками.
    
    \vspace{28 pt}
    \hspace{15 pt} 
    \large \textbf{Термодинамика} 
    
    Уравнение Менделеева~---Клайперона~--- уравнение состояния идеального газа, имеющее вид $pV = \nu RT$, где $p$~--- давление, $V$~--- объем, занимаемый газом, $T$~--- температура газа, $\nu$~--- количество вещества газа, а $R$~--- универсальная газовая постоянная.
    
    \vspace{28 pt}
    \hspace{15 pt}
    \large \textbf{Геометрия  \hfill Планиметрия}
    
    Для \textit{плоского} треугольника со сторонами $a$, $b$, $c$ и углом $\alpha$, лежащим против стороны $a$, справедливо соотношение \[a^2 = b^2 + c^2 - 2bc\cos\alpha,\] из которого можно выразить косинус угла треугольника: \[\cos\alpha = \frac{b^2 + c^2 - a^2}{2bc}.\]
    
    
    Пусть $p$~--- полупериметр треугольника, тогда путем несложных преобразований можно получить, что \[\tg\frac{\alpha}{2}= \sqrt{\frac{(p - b)(p - c)}{p(p - a)}},\] На сегодня все, пожалуй, хватит... Удачи!
\end{document}
    

   
   
\end{document}